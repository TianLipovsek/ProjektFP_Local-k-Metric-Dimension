\documentclass[a4paper,12pt]{article}
\usepackage[slovene]{babel}
\usepackage[utf8]{inputenc}
\usepackage[T1]{fontenc}
\usepackage{lmodern}
\usepackage{amsmath}
\usepackage{amssymb}
\usepackage[shortlabels]{enumitem}
\usepackage{graphicx}

\newtheorem{definition}{Definicija}

\pagestyle{plain}

\begin{document}
\author{Tian Lipovšek in Klara Gruden}
\date{December 2023}
\title{Lokalna k-matrična dimenzija grafov - kratek opis}
\maketitle


\section{Opis}
Following the paper [10], implement an ILP model for this invariant, and then write separate
small programs in Sage to answer each of following questions by exhaustive search.
\begin{enumerate}
    \item Determine $ldim_k(G)$ for paths, cycles, complete graphs, bipartite complete graphs, hypercubes and some 
    other interesting classes of graphs and try to guess the possible formulas based on the computations.
    \item Try to determine graphs G which satisfy $ldimk(G) = dimk(G)$ for a given $k$ for $k=1,2,3,...$.
    For small graphs, apply a systematic search; for larger ones, apply some stochastic search.
\end{enumerate}

\section{Definicije}

    \begin{definition}
        Vozlišče $s$ reši par vozlišč $x$, $y$ v grafu $G$, če $d(s,x) \neq d(s,y)$.
    \end{definition}

    \begin{definition}
        Rešujoča množica grafa $G$ je množica $S$, za katero velja, da za vsaki dve 
        vozlišči $x$ in $y$ v $V(G)$ obstaja vozlišče $s \in S$, ki reši par vozlišč $x$,$y$.
    \end{definition}

    \begin{definition}
        Metrična dimenzija povezanega grafa $G$, označena z $dim(G)$, je definirana kot velikost najmanjše 
        množice $S \subseteq V(G)$, ki razlikuje vse pare vozlišč v $G$.
    \end{definition}     

    \begin{definition}
        Lokalna rešujoča množica grafa $G$ je množica $S$, za katero velja, da za vsaki sosednji 
        vozlišči $x$ in $y$ v $V(G)$ obstaja vozlišče $s \in S$, ki reši par vozlišč $x$,$y$.
    \end{definition}

    \begin{definition}
        K-metrična dimenzija, označimo jo s $dim_k(G)$ je metrična dimenzija, 
        pri kateri hočemo, da za vsak par vozlišč obstaja vsaj k vozlišč $s \in S$, ki rešijo par vozlišč $x$,$y$.
    \end{definition}   
    
    \begin{definition}
        Lokalna metrična dimenzija, označena z $ldim(G)$, predstavlja moč najmanjše 
        lokalno rešujoče množice grafa $G$.
    \end{definition} 

\section{Problem} 
V lokalni k-metrični dimenziji želimo najti najmanjšo množico vozlišč, ki omogoča razlikovanje sosednjih 
vozlišč v grafu. To se naredi tako, da za vsak par sosednjih vozlišč x, y obstaja vozlišče s v množici, 
pri čemer velja $d(s, x) \neq d(s, y)$. Cilj je določiti ldimk(G) za različne razrede grafov in preučiti, 
ali obstajajo grafi, kjer lokalna k-metrična dimenzija sovpada z običajno k- metrično dimenzijo.    


\section{Načrt dela}
    \begin{itemize}
        \item Napisala bova celoštevilski linearni program, s katerim bova s pomočjo simulacij, 
        za različne vrste grafov poskušala uganiti formule za lokalno k-metrično dimenzijo
        \item Poskušala bova ugotoviti, ali obstajajo grafi G, kjer lokalna k-metrična dimenzija sovpada z 
        običajno k-metrično dimenzijo za dano vrednost k.
    \end{itemize}

    K prvemu delu naloge bova pristopila tako, da bova v Sage prepisala naslednji CLP:
    Naj bo $V(G)=\left\{v_{1}, \ldots, v_{n}\right\}$ množica vozlišč grafa $G$. 
    Naj bo $S \subseteq V(G)$. 
    Definiramo celoštevilske spremenljivke $x_{i}$ za vsako vozlišče grafa po sledečem predpisu 
    
    $$
    x_{i}= \begin{cases}1 & \text { if } v_{i} \in S \\ 0 & \text { if } v_{i} \notin S\end{cases}
    $$
    
    ter spremenljivko $y_{ij}$ kot

    $$
    y_{ij}= \begin{cases}1 & \text { if } v_{i} \neq v_{j}  \\ 0 & \text { if } v_{i} = v_{j}\end{cases}
    $$

    CLP za iskanje lokalne k-metrične dimenzije bo torej izgledal tako:

    $$
    \begin{array}{ll} 
    & \min \sum_{i=1}^{n} x_{i} \\
    \text { s.t. } & \sum_{i,j} y_{ij} \geq 1; \text {za vsa sosednja vozlišča i,j}\\
    & \sum_{j} y_{ij} \geq k \cdot x_{i}; \text {za vsa vozlišča i}\\
    & x_{i} \in\{0,1\} \text { za vse } 1 \leq i \leq n \\
    & y_{j} \in\{0,1\} \text { za vse } 1 \leq j \leq n
    \end{array}
$$
\end{document}